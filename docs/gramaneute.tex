\documentclass[12pt]{article}
\usepackage[utf8]{inputenc}
\usepackage[catalan]{babel}
\usepackage[margin=3cm]{geometry}
\usepackage{parskip}
\usepackage{hyperref}
\usepackage{lmodern}

\title{El Gramaneute\\\large Figura funcional i narrativa per a l’orquestració de models de llenguatge}
\author{Dídac Valenciano Gener}
\date{Maig de 2025}

\begin{document}
	
	\maketitle
	
	\begin{abstract}
	\noindent	Aquest text, concebut originalment com un \textit{prompt} funcional per a generar l’estructura d’un script Python, ha esdevingut una figura narrativa que defineix i humanitza el procés d’orquestració de models de llenguatge dins el \textbf{Projecte Assimov}. El \textbf{Gramaneute}, o \textit{bon pastor (de llm's)}, simbolitza la interfície entre la intenció humana i la distribució algorítmica dels rols funcionals en un sistema multimodel.
	\end{abstract}

\section*{El Gramaneute, o "bon pastor" (de llm's) — Prompt per indicar l'estructura d'un script de python.}
	
	Un "bon pastor" escolta el prompt de l'usuari, i envia aquest prompt a un llm, el qual li suggerix una definició dels rols funcionals, i tria de llm's de cada rol, en funció del prompt original.
	
	Quins models s'hi ajusten més a cada un, és una cosa que el bon pastor tria en funció de les possibilitats, físiques i econòmiques de l'usuari.
	
	El "bon pastor" mostra a l'usuari un resum de com es durà a terme l'empresa sol·licitada en el prompt original. I si l'usuari no ho ha indicat, el "bon pastor" li indica quina quantitat de seccions de primer nivell es preveu que hi hagi i quantes pàgines per secció, i en total a l'usuari, segons el que el "bon pastor" ha trobat adient, així com el que li han negociat els diversos llm de rols funcionals.
	
	L'usuari pot fer algun retoc, per ajustar la situació al seu gust, i indicar "seguir" amb el procés; o parar si així fos tant necessari.\\
	
	\medskip
	
	Al següent pas el "bon pastor" contacta de nou amb un llm amb la informació anterior, i aquest li retorna uns yaml's generals del text amb ambientació, target, estil narratiu, així com una yaml per cada secció de primer nivell on s'hi detallen títol, l'estructura completa de l'esquelet d'aquella secció de primer nivell (amb tots els títols) així com temes, frases o punts clau, habilitats específiques (gràfics, taules, pies, dibuixos DALL·E, escenes de sexe, ...) objectiu que s'aconsegueix, etc, coses que proporcionen els distints rols abans esmentats.
	
	El "bon pastor" els ensenya a l'usuari. El "bon pastor" sempre demana la llengua de sortida del text a l'usuari, si aquest no ho ha fet abans, per evitar sorpreses.\\
	
		\medskip
		
	Una vegada revisat per l'usuari tota aquesta estructura, i fet algun retoc pertinent per a agradar més a l'usuari, el "bon pastor" demana les claus dels distints llocs on hi ha els "bons llm's" amb una aproximació del cost i la durada que tendrà aquest procés.
	
	Si les claus ja eren introduïdes amb anterioritat, es limitarà a demanar permís i a informar del cost aproximat. Ni l'usuari ni el "bon pastor" volen un sistema local, perquè ja fa prou calor a Mallorca i l'energia aquí és cara.\\
	
	\medskip
	
	El "bon pastor" suggereix que si s'usen caches pot augmentar el preu, però millora la qualitat, i que cap la possibilitat de fer feina en paral·lel sense perdre massa qualitat, i per tant ser més ràpid, però no més bo.
	
	L'usuari tria com s'estima més que es cuini la nostra empresa, amb aquestes opcions de temps, costos afegits, rendiments i finura de resultats.
	
	El "bon pastor" envia els corresponents yaml's a les caches, amb el beneplàcit de l'usuari, i s'encarrega d'orquestrar la concurrència i/o seqüència entre els llm's i l'escriptura en el fitxer .tex dels diferents fragments de contingut que, finalment, li fa arribar un "bon traductor" i/o "estilista purista" per si cal cap retoc final.\\
	
	\medskip
	
	En acabar l'usuari, fet de carn i ossos, rep de la mà del "bon pastor", fet de codi python, un document .tex. S'encaixen les mans i es despedeixen.
	
	\vspace{1cm}
	
	\noindent\rule{\textwidth}{0.4pt}
	
	\vspace{0.5cm}
	
	\noindent
	\textit{Aquest text està llicenciat sota una Llicència Creative Commons Reconeixement – No Comercial – Compartir Igual 4.0 Internacional (CC BY-NC-SA 4.0).}\\
	\url{https://creativecommons.org/licenses/by-nc-sa/4.0/deed.ca}
\end{document}
 
