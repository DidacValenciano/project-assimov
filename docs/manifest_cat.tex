\documentclass[12pt,a4paper]{article}
\usepackage[utf8]{inputenc}
\usepackage[catalan]{babel}
\usepackage[margin=2.5cm]{geometry}
\usepackage{inconsolata} % Font monoespaiada agradable
\usepackage[dvipsnames]{xcolor} % Colors macos
\usepackage{tcolorbox}
\usepackage{hyperref}
\hypersetup{colorlinks=true, linkcolor=MidnightBlue, urlcolor=MidnightBlue}
\usepackage{fancyvrb} % per mostrar codi

\definecolor{mykeyword}{rgb}{0.0, 0.5, 0.0}       % verd per a keywords
\definecolor{mytype}{rgb}{1.0, 0.65, 0.0} % groc taronja fort


\definecolor{myfunction}{rgb}{0.0, 0.4, 0.7}       % blau per a noms de funció
\definecolor{myconst}{rgb}{0.6, 0.1, 0.1}          % vermell fosc per a constants
% Bloc de codi configurat amb colors
\usepackage{listings}
\lstset{
	inputencoding=utf8,
	extendedchars=true,
	literate=
	{à}{{\`a}}1
	{è}{{\`e}}1
	{é}{{\'e}}1
	{í}{{\'i}}1
	{ï}{{\"i}}1
	{ò}{{\`o}}1
	{ó}{{\'o}}1
	{ú}{{\'u}}1
	{ü}{{\"u}}1
	{ç}{{\c{c}}}1
	{ñ}{{\~n}}1
	{Á}{{\'A}}1
	{É}{{\'E}}1
	{Í}{{\'I}}1
	{Ï}{{\"I}}1
	{Ó}{{\'O}}1
	{Ú}{{\'U}}1
	{Ü}{{\"U}}1
	{¿}{{?``}}1
	{¡}{{!``}}1
}
\usepackage{tcolorbox}

\tcbuselibrary{listings, skins, breakable}


\newtcblisting{metaoutput}{
	listing only,
	title=github.com/nassaba/project-assimov - AI-enhanced content generation pipeline,
	coltitle=black,
	breakable,
	enhanced,
	colback=white,
	colframe=gray!80!black,
	sharp corners,
	listing options={
		basicstyle=\ttfamily\small,
		keywordstyle=\color{mykeyword}\bfseries,
		stringstyle=\color{BurntOrange},
		commentstyle=\color{gray},
		identifierstyle=\color{black},
		emph={phase_0_prompt_intake, phase_1_generate_structure,
			phase_2_expand_sections, phase_3_refine_coherence,
			phase_4_compile_tex, phase_5_final_review},
		emphstyle=\color{myfunction},
		emph={[2]Path, plan, tex, yamls},
		emphstyle={[2]\color{mytype}},
		emph={[3]True, False, None},
		emphstyle={[3]\color{myconst}},
		morekeywords={import, from, def, return, if, else, class, for, while},
		numbers=left,
		numberstyle=\tiny\color{gray},
		numbersep=5pt,
		xleftmargin=2.2em,
		breaklines=true
	}
}

\definecolor{mygray}{gray}{0.45}
\definecolor{mygreen}{rgb}{0,0.5,0}
\definecolor{myorange}{rgb}{0.8,0.4,0}

\title{}
\author{}
\date{}

\begin{document}


\noindent
\ttfamily
\textcolor{mygreen}{\textbackslash documentclass}\textcolor{black}{[12pt,a4paper]\{article\}}\\
\textcolor{mygreen}{\textbackslash usepackage}\textcolor{black}{[utf8]\{inputenc\}}\\
\textcolor{mygreen}{\textbackslash usepackage}\textcolor{black}{[catalan]\{babel\}}\\
\textcolor{mygreen}{\textbackslash usepackage}\textcolor{black}{[margin=2.5cm]\{geometry\}}\\
\textcolor{mygreen}{\textbackslash usepackage}\textcolor{black}{\{inconsolata\}} \textcolor{mygray}{\%\ Font monoespaiada agradable}\\
\textcolor{mygreen}{\textbackslash usepackage}\textcolor{black}{\{hyperref\}}\\
\textcolor{mygreen}{\textbackslash hypersetup}\textcolor{black}{\{colorlinks=true, linkcolor=MidnightBlue, urlcolor=MidnightBlue\}}\\
\textcolor{mygreen}{\textbackslash usepackage}\textcolor{black}{\{fancyvrb\}} \textcolor{mygray}{\%\ per mostrar codi}

\medskip

\noindent\textcolor{mygreen}{\textbackslash title}\textcolor{black}{\{PROJECTE ASSIMOV: Un Manifest per a Educadors en l'Era dels Transformers\}}\\
\textcolor{mygreen}{\textbackslash author}\textcolor{black}{\{Dídac Valenciano Gener\}}\\
\textcolor{mygreen}{\textbackslash date}\textcolor{black}{\{maig del 2025\}}

\medskip

\noindent\textcolor{mygreen}{\textbackslash begin}\textcolor{black}{\{document\}}\\
\textcolor{mygreen}{\textbackslash section*}\textcolor{black} \{ \sffamily L’any 2025 és —i serà— l’any ChatGPT\}.\\
				
				\bigskip
				
				A l'Europa ibèrica i mediterrània dins la UE, amb un subconscient meca latent que avui encara es nota, \emph{Mazinger Z} va deixar una profunda empremta cultural, emocional i en diversos altres àmbits que explica alguns comportaments compartits de la \emph{Generació X}. Si tens entre 45 i 60 anys, ja has pensat en “\emph{¡Puños fuera!}”, has recordat com vas sobreviure al \emph{Blandiblú}, als \emph{Clicks} de \emph{Famobil}, i a veure dibuixos violents sense convertir-te en un psicòpata.
				
				\medskip
				
				Als Estats Units, \emph{Hasbro} inicia el relleu generacional amb la saga \emph{Transformers}, que culmina culturalment l’any 2007 quan, just arrencar el metratge, els \emph{Decepticons} ataquen una base militar nord-americana al desert de \emph{Qatar}. Mentrestant, \emph{Megan Fox} i \emph{Shia LaBeouf} ens deixen sense alè. És en aquest moment quan una nova imatge s’enganxa al subconscient d’una altra generació: un petit \emph{Decepticon} —un \emph{transformer}, però dels dolents—, amagat entre les pertinences dels protagonistes, absorbeix metall de l’entorn per modificar-se, estirant les seves formes, perllongant-se i adaptant-se fins a assolir el seu objectiu.
				
				\medskip
				
				Deu anys després, fruit d’un llarg recorregut d’investigació per part de múltiples entitats acadèmiques i privades, en paral·lel al món de l’entreteniment, \emph{Google} presenta públicament l’any 2017 una tecnologia algorítmica que es replica, s’estira i amplia la seva pròpia estructura, modificant-se i perllongant-se fins que, de manera purament probabilística, aconsegueix assolir el seu objectiu. Quina sorpresa —i quina coincidència— que aquesta tecnologia, desconeguda fins aleshores, rebés el nom de: \emph{transformers}.
				
				\medskip
				
				\textit{"Attention is All You Need"}, publicat per \emph{Google} el 2017, va crear rebombori —especialment a \emph{Twitter} (o \emph{Grok}, o com es digui aquesta setmana). Però no va ser fins que \emph{OpenAI} va agafar el relleu que ens van col·locar al davant, literalment, un petit \emph{transformer}. Un que, cada vegada que li parlam, \emph{tokenitza} el nostre llenguatge, i com aquell \emph{Decepticon} al desert de \emph{Qatar}, s'estira, s'adapta i reorganitza la seva forma per generar, de manera purament probabilística, una resposta que optimitzi la situació que té al davant.
				
				\medskip
				
				I la gràcia, amics meus, és que ho simula tan bé, que si entenem per \emph{intel·ligència artificial} una simulació funcional d’intel·ligència... bé, potser hi ha altres implementacions, però aquesta —definitivament— n’és una.
				
				\medskip
				
				El que ve tot seguit és una altra implementació. Una que, com els algoritmes d’\emph{OpenAI}, s’estira, s’adapta i reorganitza la seva pròpia estructura per generar, de manera purament probabilística, una resposta òptima a la situació que té al davant: crear de manera eficient contingut didàctic, un article, un llibre. Tot plegat perquè nosaltres, \emph{professors}, puguem disposar de les nostres pròpies armes per a la batalla que ens espera.
				
			
			
			\section*{\texttt{AI-enhanced LaTeX generation pipeline}}
			
			\begin{metaoutput}
# Phase 0: Interpret prompt and assign roles with gramaneute + llm_router

# Phase 1: Generate structural skeleton with yaml_generator using Claude or o3 or ...

# Phase 2: Expand each section with role-assigned LLMs (GPT-4o, MythoMax, Lit-6B...) using cached YAMLs

# Phase 3: Review narrative coherence, argument and tone with o3 or Claude using section context

# Phase 4: Compile formatted output into .tex with writer or translator

# Phase 5 (optional): Final stylistic polish by purist LLM or human reviewer

# CONFIGURATION
# Load API key (replace with your secure method)

import openai
from pathlib import Path
import time

				
# AI-enhanced content generation pipeline (Assimov)

def phase_0_prompt_intake():
    """Receive user prompt and determine functional roles + LLMs."""
    prompt = user_input()
    roles = assign_roles(prompt)  # via gramaneute.py -> llm_router.py
    llms = select_models(roles, config="config.yaml")
    return plan(roles, llms)

def phase_1_generate_structure(plan):
    """Create YAML skeleton per section with titles, themes, targets."""
    yamls = []
    for section in plan.sections:
    yamls.append(generate_yaml(section))  # via yaml_generator.py or real LLM
    return yamls

def phase_2_expand_sections(yamls):
    """Expand each section via assigned LLM, using cached context."""
    content = []
    for yaml in yamls:
        llm = yaml.assigned_model
        section_text = expand_from_yaml(yaml, llm=llm, use_cache=True)
        content.append(section_text)
    return content

def phase_3_refine_coherence(content):
    """Polish narrative flow, argument structure, and tonal coherence."""
    polished = []
    for section in content:
        coherent = enforce_coherence(section)  # coherence supervisor
        refined = polish_argument(coherent)
        toned = adjust_tone(refined)
        polished.append(toned)
    return polished

def phase_4_compile_tex(polished):
    """Format polished content into a .tex document with structure."""
    doc = initialize_tex()
    for section in polished:
        doc.append(format_section(section))  # via writer.py or translator.py
    return doc

def phase_5_final_review(tex_document):
    """Optional: stylistic and poetic pass by a purist LLM or human."""
    reviewed = manual_review(tex_document)
    return reviewed

# Main pipeline execution

if __name__ == "__main__":
    plan = phase_0_prompt_intake()
    yamls = phase_1_generate_structure(plan)
    raw_content = phase_2_expand_sections(yamls)
    refined = phase_3_refine_coherence(raw_content)
    tex = phase_4_compile_tex(refined)
    final_output = phase_5_final_review(tex)
    save(final_output, "output/document.tex")
				
			\end{metaoutput}
			
		\end{document}
