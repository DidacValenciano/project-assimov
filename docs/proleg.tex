\documentclass[12pt,a4paper]{article}
\usepackage[utf8]{inputenc}
\usepackage[catalan]{babel}
\usepackage[margin=2.5cm]{geometry}

\usepackage{titlesec}
\usepackage{hyperref}

\titleformat{\section}{\large\bfseries}{\thesection.}{1em}{}
\titleformat{\subsection}{\normalsize\bfseries}{\thesubsection.}{1em}{}


\begin{document}
	
	\section*{Pròleg a l'educació que ha de venir}
	
	\subsubsection*{La falsa promesa d’un futur igualitari}
	
	A \textit{La màquina del temps} de H.G. Wells, la humanitat es divideix en dues espècies: els \textit{Eloi} (fràgils, ignorants, feliçment desconnectats) i els \textit{Morlocks} (durs, tècnics, explotats). És fàcil pensar que és ciència-ficció, però és només una metàfora \textbf{massa fidel del que passa quan el coneixement es concentra i es tanca.}
	
	La divisió entre “els que poden” i “els que serveixen” \textbf{no és cap invenció del segle XIX.} És el fonament estructural de tota societat jerarquitzada des que existeix l’escriptura i la propietat.
	
	A Egipte, Mesopotàmia, Grècia o Roma, la desigualtat no era una anomalia: era l’esquema. L’alfabetització era un privilegi; la majoria treballava, callava o servia. Els coneixements es concentraven, s’heretaven, es certificaven. El poder s’exercia mitjançant el control del saber.
	
	Res no ha canviat. Només les eines.
	
	On abans hi havia tauletes de fang i escribes, ara hi ha interfícies neuronals i accessos restringits. On abans hi havia títols nobiliaris o ordres religioses, ara hi ha màsters privats, certificacions exclusives i ecosistemes tancats d’aprenentatge digital.
	
	\textbf{La IA no ha democratitzat res per si sola. I el sistema educatiu encara no ha reaccionat.}
	
	Si no obrim aquest coneixement crític des de l’escola pública —si no ens anticipam— tornarem a generar una elit de formats; i una massa de servents muts seguint les màquines.
	
	\textbf{Això no és el futur. Això és ara.}
	
	I si no fem res, \textbf{el forat s’eixamplarà.}  
	No tindrem IA per tothom. Tindrem \textbf{dues humanitats:} una amb \textit{prompts}, l’altra amb ordres.
	
	Aquest coneixement crític no ha de començar només amb tecnologia. \textbf{Ha de començar amb consciència.}
	
	\bigskip
	
	\subsubsection*{El sostre invisible de la humanitat}
	
	Des de fa més de cinc mil anys, la humanitat escriu sobre una idea persistent: donar vida a allò que no en té, crear amb paraules, fang, metall o símbols una entitat capaç de decidir per si mateixa. 
	
	Als temples de Mesopotàmia ja es modelaven servents de fang amb funcions assignades. Als relats mítics, déus i deesses programaven éssers amb \textit{me}, codis de comportament sagrats. A Grècia, Pigmalió donava forma a l'amor ideal en marbre. Al segle XIX, Mary Shelley creava un monstre amb consciència. I Carlo Collodi escrivia la metàfora perfecta: Pinotxo, una fusta viva que desitja ser real.
	
	Durant mil·lennis, aquesta idea s’ha alimentat amb mites, llegendes, literatura i, més tard, pel·lícules. \textbf{No només com a ficció, sinó com a fita.} Com si la intel·ligència artificial fos una mena de \textbf{destí narratiu}. Alguna cosa que tard o d’hora arribaríem a construir. I així ho hem fet.
	
	Ara que la tenim —en forma de models de llenguatge, algoritmes generatius i agents digitals capaços de respondre, conversar, simular i influir—, ens trobam davant un problema més profund que cap dilema tecnològic:  
	\textbf{Què passa quan l'espècie arriba allà on sempre ha volgut arribar?}
	
	Igual que l’exploració de l’espai —després del mar, les muntanyes i els cels—, \textbf{la IA marca un dels dos límits simbòlics que la nostra espècie havia imaginat com a finals.} I la nostra cultura, sense saber-ho, s’ha preparat per això... però no per al que vindrà després.
	
	Aquesta crida a l’aula no pretén sumar una eina, sinó evitar que el llenguatge deixi de ser nostre. Pretén obrir una porta al futur —un que ja és present— i dotar els joves de la capacitat d’habitar-lo amb consciència, llenguatge i criteri. \textbf{Perquè la IA ha deixat de ser metàfora. Ja no és monstre, ni mirall, ni profecia. És una interfície. I cal saber com parlar-hi.}
	
	Si la societat no ho entén, és perquè encara està processant un trauma col·lectiu:  
	\textbf{hem arribat a la meta que la literatura ens havia venut com a inassolible... i no hi havia res més enllà.}
	
	Aquest relat comença aquí. En el lloc exacte on el sostre... cau.  
	I ara, potser per primera vegada, sabent per què ha caigut, podem decidir si hi entra llum o pols.\\
	
	\subsubsection*{Epíleg, "Pacífic i amb més Rim" (mallorquina)}
	\noindent Quan es cervell fa germanor, \\
	raó i símbol fan dança;  \\
	s’alça un pont lliberador  \\
	i la nova ona avança.  \\
	
	\noindent Amb l’art fi de fer es «prompting»\\  
	l’alumne pensa, no torba;  \\
	creua memòria i destí,  \\
	ja mai més cap mur l’estorba.\\  
	
	\noindent Si ben units obrim la ment,  \\
	la llum venç tota ferida;  \\
	es llenguatge és l’armament  \\
	d’una escola ben teixida. \\

\section*{Nota editorial}

Aquest "pròleg"~s’ha concebut com un text independent dins un projecte pedagògic no institucional, impulsat de manera autònoma i crítica des de l’àmbit docent. No forma part de cap programa curricular oficial, tot i haver estat redactat amb esperit de servei públic i compromís educatiu.

S’ofereix a mitjans culturals, educatius i de reflexió crítica com a peça d’opinió per obrir el debat sobre el paper del llenguatge i la consciència pedagògica en l’era de la intel·ligència artificial.

\section*{Llicència d’ús}

Aquest text es distribueix sota una llicència \textbf{Creative Commons Reconeixement – NoComercial – CompartirIgual 4.0 Internacional (CC BY-NC-SA 4.0)}. Això vol dir que pots copiar-lo, redistribuir-lo i adaptar-lo sempre que:

\begin{itemize}
	\item se’n reconegui l’autoria (Dídac Valenciano Gener),
	\item no se’n faci un ús comercial sense permís explícit,
	\item i qualsevol obra derivada es comparteixi amb la mateixa llicència.
\end{itemize}
 
És a dir, el text pot ser adaptat per lectures públiques, vídeos o àudios, sempre que se’n respecti l’autoria i esperit original.\\
 
Més informació: \url{https://creativecommons.org/licenses/by-nc-sa/4.0/deed.ca}


	\section*{Nota de l’autor}
	
\textbf{Aquest "pròleg"~s’ha escrit a dues veus:} una inspirada en el lòbul esquerre (lògic, estructurat, discursiu) i l’altra en el dret (simbòlic, històric, emocional). La seva fusió final, a l’epíleg, recupera la metàfora de “Pacific Rim” —franquícia de ciència-ficció on dues ments han de sincronitzar-se per moure una sola acció col·lectiva— per suggerir que només la coordinació entre hemisferis —o discursos— pot activar una acció educativa coherent i compartida.\\

El títol “amb més Rim” afegeix, a més, un segon nivell de lectura local: en mallorquí, tenir “rim” és tenir ritme, glosa i picada d’ullet improvisada. I això —aquest toc de poble, de llengua viva i d’ironia cultivada— també forma part del que volem salvar.

\textbf{Perquè el perill no és la IA. És la passivitat}. És creure que OpenAI, Meta o Google et respectaran la llengua si tu mateix no ho exigeixes.

La IA no salvarà el català. Però pot ajudar-lo a sobreviure — si decidim entrenar-la perquè ho faci. \textbf{Posem "prompt" allà on s'hi espera submissió.}\\

	
	 Tot i que alguns ajustos d’estil han estat assistits per ChatGPT, la construcció sim\-bò\-li\-ca, l’analogia clàssica i la seqüència narrativa són humanes, intencionades i deliberades. ChatGPT pot ajudar, sí, però encara no entrellaça metàfores amb estratificació històrica, ciència-ficció i crítica institucional alhora. Ni apel·la a la unitat per fer front a nous reptes. Ni a la sàtira en clau sci-fi que amaga el rerefons d'aquest assaig.
	
	\textbf{No vos flipeu —com deim en mallorquí—. Encara queda molt per ensenyar-li.}
	I potser, ben mirat, el \emph{prompt} per provocar aquest tipus de text diu tant com el text mateix.
	
	\bigskip
	
	\noindent
	\textbf{Dídac Valenciano Gener}  \\
	Professor d’àmbit científico-tècnic.\\
	IES Guillem Cifre de Colonya\\
	\texttt{didac@valenciano.cat} \\
	Telèfon: +34 626 748 148
\end{document}
